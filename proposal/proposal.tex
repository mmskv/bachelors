\RequirePackage[OT1]{fontenc}
\documentclass[conference]{IEEEtran}

\IEEEoverridecommandlockouts
\usepackage[OT1]{fontenc}
%\usepackage{cite}
\usepackage{amsmath,amssymb,amsfonts}
\usepackage{algorithmic}
\usepackage{graphicx}
\usepackage{textcomp}
\usepackage{xcolor}
\usepackage{flushend}
\def\BibTeX{{\rm B\kern-.05em{\sc i\kern-.025em b}\kern-.08em
    T\kern-.1667em\lower.7ex\hbox{E}\kern-.125emX}}

\usepackage[backend=biber, style=numeric-comp, language=auto, autolang=other, sorting=none]{biblatex}
\addbibresource{references.bib}

\begin{document}

\title{Resiliently Synchronizing SFN Networks: Combining Precise Time Signals from GPS and Longwave Radio Stations}

\author{\IEEEauthorblockN{Maxim Emelyanenko}
\IEEEauthorblockA{\textit{Department of Information Security} \\
\textit{National Research University ``Higher School of Economics''}\\
Moscow, Russia \\
mvemelyanenko@edu.hse.ru}}

\maketitle

\begin{abstract}
This study presents a novel approach for enhancing the resilience of time
synchronization in Single Frequency Networks (SFN), particularly within the
context of digital terrestrial television broadcasting standards. It
critically assesses the reliance on Global Navition Satellite Systems
(GNSS) for SFN synchronization, identifying it as a potential single point
of failure due to GNSS's susceptibilities, such as ephemeris errors,
ionospheric delays, solar activity, and potential attacks that compromise
signal accuracy and reliability. The paper explores the potential of
eLoran, a system that was developed to provide resiliency for global
navigational systems by complementing GNSS systems with ground-based time
signal radio stations for trilateration. Development of a similar system is
proposed advocating for a holistic approach that combines GNSS with
existing terrestrial time signal radio station infrastructure to enhance
SFN resilience against vulnerabilities.
\end{abstract}

\begin{IEEEkeywords}
SFN, GNSS, eLoran, DVB-T, time synchronization
\end{IEEEkeywords}

\section{Introduction}

\IEEEPARstart{W}{ith} single frequency networks evolving to meet
ever-increasing throughput demands,  the precision of timing synchronization
becomes crucial for the network performance. The advent of Digital Video
Broadcasting -- Terrestrial (DVB-T) standard and their subsequent enhancements
in DVB-T2 have significantly raised the bar for synchronization accuracy. These
standards, employing orthogonal frequency-division multiplexing (OFDM)
modulation, offer substantial data througput but also introduce strict
requirements for transmission synchronization to prevent inter-symbol
interference and optimize bandwidth utilization~\cite{ETSIDVBT22009,liang2007timing}. Historically, GNSSs, such as GPS and GLONASS, have been the
cornerstone for achieving this level of synchronization across the
geographically distributed infrastructure of SFNs because other systems like
Precision Time Protocol (PTP), while effective in certain contexts with
consistent network latencies, like a localized data center, encounter scaling
challenges over geographical expanses, rendering them unsuitable~\cite{ieee1588_2008,ptpscale}. However,
the reliance on GNSS without any complementary or backup system introduces a
range of vulnerabilities, including susceptibility to ephemeris errors,
ionospheric delays, solar activities, and potential security threats like
jamming and spoofing which are thoroughly assessed in the section~\ref{gnss-vuln}. These
vulnerabilities not only compromise the integrity and reliability of broadcast
services but also highlight a critical single point of failure within the
synchronization framework of SFNs.

To mitigate similar risks in the context of global navigation, the eLoran
system was developed as a terrestrial-based alternative to GNSS, designed to
offer a robust solution to the single point of failure issue by providing a
complementary source of timing signals usable for positional trilateration~\cite{eloran}.
The approaches suggested by eLoarn are further reviewed in this paper in the
section~\ref{eloran-chap}.

Inspired by the principles of eLoran, this study explores the feasibility of
leveraging terrestrial radio stations, specifically those managed by the
Russian State Service for Time, Frequency, and Earth Rotation Parameters,
alongside similar stations in Germany, the United Kingdom, and Japan~\cite{itur1997standard}.
These stations, characterized by their picosecond-precision in time signal
dissemination, present a viable solution for enhancing SFN synchronization
resilience and accuracy.

This paper proposes a hybrid synchronization model that integrates the
precision of GNSS with the reliability of ground-based time signals. By
examining fitness of the existing infrastructure and evaluating the potential
of these terrestrial time signal sources, the study seeks to evaluate the
scalability limitations inherent in ground-based systems and to outline
effective algorithms for the detection and integration of these signals within
the SFN synchronization framework. The exploration of this hybrid model
provides a way for a strategic approach to safeguarding digital broadcasting
networks against the multifaceted spectrum of threats impacting GNSS
reliability.

\section{Literature review}

\subsection{GNSS Vulnerabilities}\label{gnss-vuln}

The vulnerabilities of GNSS encompass a multifaceted array of challenges that
significantly impact the accuracy and reliability of signal synchronization,
crucial for the optimal operation of SFN. These vulnerabilities manifest
through a variety of technical and natural phenomena. Technical issues such as
ephemeris errors and time system inaccuracies can introduce significant
discrepancies in positioning and timing signals~\cite{dempster2001vulnerable}.
Additionally, natural phenomena like ionospheric and tropospheric signal delays
present unpredictable challenges, distorting the signals transmitted by
satellites. The impact of solar activity further complicates this scenario,
inducing additional signal interference and
delays~\cite{infrastructure2001vulnerability}.

Moreover, GNSS vulnerabilities extend beyond natural and technical challenges,
encompassing serious threats from intentional interference and attacks, such as
jamming and spoofing. Jamming involves the disruption or obstruction of GNSS
signal reception, while spoofing entails the broadcasting of false signals to
mislead GNSS receivers. Instances of such attacks, including documented cases
of GPS signal jamming by North Korea affecting South Korea, highlight the
increasing prevalence and sophistication of these threats~\cite{SeoKim2013}.
The low power of satellite-borne signals makes GNSS particularly susceptible to
such attacks, underscoring the need for robust countermeasures.

SFNs, relying heavily on precise timing synchronization, are especially
vulnerable to the ramifications of GNSS disruptions. These vulnerabilities
spotlight the critical importance of developing and implementing alternative
time signal acquisition methods to ensure higher levels of security and
reliability in systems dependent on accurate time synchronization. Reference
\cite{machaj2021impact} underlines the serious consequences of GNSS
interferences and attacks, advocating for the exploration and adoption of
alternative approaches to bolster the resilience and reliability of such
critical infrastructure. This necessitates a comprehensive understanding of
GNSS vulnerabilities and a proactive approach to safeguarding SFN
synchronization against the myriad of threats facing GNSS, ensuring the
continuity and integrity of digital broadcasting services.

\subsection{SFN Synchronization}\label{sfn-sync}

Single Frequency Networks rely on the precise synchronization of
broadcast signals across multiple transmitters to ensure seamless signal
delivery to receivers within the network. This synchronization is crucial for
preventing signal interference and maximizing network efficiency, as it allows
multiple transmitters to broadcast the same content simultaneously on the same
frequency. Without accurate synchronization, signals from different
transmitters could overlap destructively, leading to signal degradation,
increased error rates, and poor reception quality. Therefore, maintaining
stringent synchronization across the SFN is essential for increasing network
throughput by decreasing length of guard intervals that account for timing
discrepancies between transmitters.

Traditionally, SFNs have relied heavily on synchronization via GNSS, including GPS and GLONASS. This method,
while widely adopted for its operational simplicity and precision, introduces a
single point of failure due to its susceptibility to various vulnerabilities
and underscores a critical need for more resilient and secure synchronization
methods.

The Integrated Time Transfer synchronization, aimed at removing GNSS dependency
within SFN networks, leverages the Time Transfer feature of the Nimbra
transport platform. This technology facilitates precise time signal
dissemination across the network, capitalizing on its inherent architecture to
maintain synchronization~\cite{Hellstrom2007}. Integrated Time Transfer system operates by
embedding time signals within the data stream transmitted across the network.
It requires a coherent network infrastructure, where each node is capable of
extracting and utilizing these embedded time signals to synchronize its
operations. Despite its innovative approach, this method encounters challenges
in universal application across all DVB-T systems. Factors such as network
configuration diversity, signal propagation delays, and the variable quality of
network links can significantly impact its effectiveness. Specifically,
environments with high network latency or those that lack uniformity in
infrastructure design may not achieve the desired synchronization precision.
While offering a level of resilience against common GNSS vulnerabilities such
as signal jamming and spoofing, the Integrated Time Transfer synchronization's
dependency on network infrastructure integrity introduces its own set of
potential weaknesses. These include susceptibility to network-based disruptions
and the need for robust, uniformly high-quality network connections to ensure
consistent synchronization performance across the SFN.

The transition towards alternative synchronization methods for SFNs is fraught
with technical, logistical, and infrastructural challenges. The establishment
of a terrestrial time signal network, for instance, requires significant
investment and strategic planning to ensure coverage, precision, and
reliability that can match or surpass GNSS-based methods. Furthermore, the
integration of multiple synchronization sources introduces new complexities in
synchronization algorithms and system design, necessitating further research
and development efforts.

\subsection{eLoran}\label{eloran-chap}

eLoran, an advanced evolution of the original Long Range Navigation (Loran)
system, stands out as a robust terrestrial-based navigational and timing
solution, designed to complement and, in certain scenarios, replace GNSS
systems like GPS. Its operational premise relies on a network of high-powered,
low-frequency time signal radio transmitters usable for trilateration that
provide wide-area coverage, capable of penetrating environments where GNSS
signals may be weak or obstructed. This characteristic, coupled with eLoran's
resistance to common forms of interference and spoofing attacks, underlines its
utility in critical navigational infrastructure.

The strength of eLoran lies in its signal's low frequency, which allows for
reliable reception even in challenging conditions, making it an ideal candidate
for secure, resilient time and frequency dissemination. This system's
architecture is designed to offer an alternative or supplementary timing
solution that mitigates the vulnerabilities inherent in satellite-based
systems, particularly in scenarios where GNSS signals are compromised due to
natural phenomena or deliberate interference.

Further development and deployment of eLoran systems across the globe,
including significant investments by countries in establishing and maintaining
eLoran infrastructure, reflect a strategic approach to safeguarding critical
communication, navigation, and timing services against the increasing threats
to GNSS reliability. The integration of eLoran into existing technological
frameworks promises enhanced operational continuity and security, making it a
cornerstone in the pursuit of GNSS-independent, resilient global positioning
networks.

\subsection{Time signal range limitations}

The investigation into the maximum achievable range for the reception of time
signals from longwave radio stations constitutes a critical research domain.
The advent of advanced Digital Signal Processing (DSP) technologies offers
promising avenues to extend the reception range, potentially facilitating
national coverage. In the context of Russia, the operation of 11 longwave
precise time signal stations hints at the possibility of achieving nationwide
coverage, as suggested by references such as the ITU-R 1997
standard~\cite{itur1997standard}. However, the vast expanse of Russia's
territory necessitates empirical validation to confirm these theoretical
capabilities.

Similar time signal radio station in Germany (DCF-77) is reported to have a
reception range extending up to 2000 km. Despite this claim, there is a notable
absence of comprehensive scientific investigations of this number. This gap in
research underscores the imperative need for dedicated experimental studies
aimed at accurately determining the maximum reception range of time signals.

\section{Methods}

The research begins with an exhaustive literature review to understand the
current landscape of time synchronization, particularly emphasizing the DVB-T
and DVB-T2 standards, Precision Time Protocol (PTP), and GNSS. This review serves to identify the state of the art
and gaps in existing research.

Exploring the propagation of low-frequency radio signals forms a significant
portion of our study. This involves investigating how these signals are
received and developing antennas with diverse characteristics aimed at ensuring
robust reception of time signals. Such an investigation is critical for
devising a reliable system that operates effectively under various conditions.

In parallel, we focus on the creation of novel Digital Signal Processing (DSP)
algorithms. These algorithms are designed to extract time signals from the
noisy data received by the antennas, a key component in achieving accurate
signal interpretation even in challenging environments.

At the heart of our experimental setup is a hardware platform developed around
the STM32 microcontroller. Selected for its performance and availability, this
platform is intended to function as a timing source for synchronizing SFN base stations.

To augment accuracy, especially during instances of signal loss, the system
incorporates oven-compensated voltage-controlled oscillators (OCVCXO). Renowned
for their stable signal performance over time, these devices are instrumental
in maintaining synchronization precision.

The performance of this comprehensive setup is rigorously evaluated through
physical tests under a variety of natural conditions and at different distances
from the radio time signal source. Such evaluations are essential for
validating the system's effectiveness in real-world scenarios.

Finally, the research develops algorithms capable of identifying discrepancies
in both GNSS and ground-based time signals. These algorithms facilitate an
internal switching mechanism between signal sources to ensure continuous
synchronization of SFN base stations.

This methodological framework underscores our dedication to advancing SFN
synchronization technology. By addressing each component of the synchronization
process, from signal reception and processing to hardware innovation, we aim to
establish a robust foundation for future developments in digital broadcasting
networks.

\section{Anticipated Results}

The anticipated outcomes of this study focus on the creation of an experimental
device or a novel synchronization method tailored for SFNs. This initiative aims to leverage longwave precise time signal stations
alongside supplementary sources to bolster resilience against GNSS
interference. The endeavor involves engineering a device or methodology that is
proficient in receiving a standard 1 PPS (pulse per second) time signal and a
10 MHz reference signal, both exhibiting minimal deviation from the national
time scale.

Field evaluations will be conducted to ascertain the maximum feasible distance
from the precise time signal radio stations' ground reference source while
still maintaining the requisite accuracy and stability. Additionally, the
project is set to devise a method for the rigorous assessment of the time
signal's accuracy and stability, facilitating its evaluation under laboratory
conditions complemented by the formulation of the necessary mathematical models
and algorithmic support.

A pivotal feature of the developed device will be its capability to seamlessly
transition between GNSS and terrestrial time signals, thereby ensuring a stable
synchronization signal, which is crucial for adherence to DVB-T2 standards. The
project aspires to achieve a precision goal of at least $10^{-10}$~Hz relative
to the Coordinated Universal Time (UTC) scale, aligning with the stringent
demands set forth for DVB-T2 broadcasting standards~\cite{Hellstrom2007}.

\section{Conclusion}

This research explores the integration of GNSS and longwave radio stations for
enhancing time synchronization in SFN networks, addressing the vulnerabilities
of GNSS systems. Through a multifaceted approach including literature review,
analysis of GNSS vulnerabilities, exploration of alternative synchronization
methods, and development of experimental devices, the study proposes a hybrid
model for SFN synchronization. This approach not only aims to mitigate the
risks associated with GNSS dependencies but also explores the potential of
ground-based time signals for improving resilience and accuracy. The
anticipated outcomes suggest promising directions for future research and
practical applications in digital broadcasting networks, emphasizing the
importance of diversifying time signal sources to ensure reliable and precise
synchronization across SFN networks.

\printbibliography

Word count: 2115

\end{document}
